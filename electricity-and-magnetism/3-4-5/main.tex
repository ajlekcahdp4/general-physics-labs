\documentclass{article}
\usepackage[utf8]{inputenc}
\usepackage[russian]{babel}
\usepackage{graphicx}
\usepackage{amsmath}
\usepackage{breqn}
\usepackage{wrapfig}
\usepackage{float}
\usepackage{multirow}
\usepackage{caption}
\usepackage{subcaption}

\graphicspath{ {./data/images} }
\author{Александр Романов Б01-107}
\title {3.4.5 Петля гистерезиса(динамический метод)}
\date{}

\begin{document}
\maketitle
\section{Введение}
\subsection{Цель работы}
Изучение петель гистерезиса различных ферромагнитных материалов в переменных полях.
\subsection{В работе используются}
Автотрансформатор, реостат, интегрирующая ячейка, амперметор, вольтметр, резистор, делитель напряжения, электронный
осциллограф, тороидальные образцы с двумя обмотками.

\section{Работа}
Запишем  некоторые характеристики образцов:
\begin{table}[H]
\centering
    \begin{tabular}{|c|c|c|c|}
        \hline
                    &Кремнистое железо  &Феррит &Пермаллой\\\hline
        $N_0$         &20                 &42     &15     \\\hline
        $N_u$         &200                &400    &300    \\\hline
        $S$, cm^2     &2                  &3      &0.66   \\\hline
        $2\pi R$, cm  &11                 &25     &14.1   \\\hline
    \end{tabular}
\end{table}
Запишем некоторые параметры установки:
\begin{table}[H]
    \centering
        \begin{tabular}{|c|c|}
            \hline
            $R_0,\; \Omega$&0.2\\\hline
            $R_u,\; k\Omega$&20\\\hline
            $C_u,\; \mu F$  &20\\\hline
        \end{tabular}
    \end{table}

\subsection{Калибровка канала X ЭО}

Закоротим обмотку \(N_0\). Ток будет синусоидален. Амперметр А подключим на измерение эффективного тока \(I_{act}\),
текущий через \(R_0\). Сигнал с этого сопротивления подаётся на вход X ЭО. В этом случае ширина горизонтальной
развёртки на экране ЭО будет соответствовать удвоеной амплитуде напряжения на \(R_0\). Измерив длину \(2x = 7\; cells\)
горизонтальной прямой на экране, ток \(I_{act} = 2.37\; A\) вычислим \(K_x\) - чувствительность канала X.

\[ K_x = \frac{2R_0\sqrt{2}I_{act}}{2x} = 0.19\; V/cell\]

\section{Выводы}

\end{document}