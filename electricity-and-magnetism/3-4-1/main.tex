\documentclass{article}
\usepackage[utf8]{inputenc}
\usepackage[russian]{babel}
\usepackage{graphicx}
\usepackage{wrapfig}
\usepackage{float}

\graphicspath{ { ./data/images/ } }
\author{Александр Романов Б01-107}
\date{}
\title{3.4.1 Диа- и парамагнетики}

\begin{document}
\maketitle

\section{Введение}

    \subsection{Цель работы}
    Измерение магнитной восприимчивости диа- и парамагнитного образцов.

    \subsection{В работе используется}
    Электромагнит, весы, милливеберметр, регулируемый источник постоянного тока, образцы.

\section{Работа}
    \subsection {Подготовка}
    Для начала снимем зависимость магнитного потока Ф, пронизывающего милливеберметр от тока I:

    \begin{table}[H]
        \centering
    \begin{tabular}{|c|c|}
        \hline
        I, amp & Ф, mWb \\\hline
        0,3  & 99     \\\hline
        0,6  & 185,3  \\\hline
        0,91 & 288,6  \\\hline
        1,20 & 372,5  \\\hline
        1,5  & 463    \\\hline
        1,81 & 545    \\\hline
        2,2  & 653,7  \\\hline
        2,6  & 735,2  \\\hline
        3,02 & 806,9  \\\hline
    \end{tabular}
    \end{table}

    Запишем параметры образцов:

    \begin {table}[H]
        \centering
    \begin{tabular}{|c|c|c|}
        \hline
                & d, cm & m, g  \\\hline
        \(Al\)  & 1,00  & 25,2  \\\hline
        \(Cu\)  & 1,00  & 83,3  \\\hline
        \(Gr\)  & 1,00  & 11,   \\\hline
        
        
    \end{tabular}
    \end{table}


    Убедившись, что весы арретированы приступим к работе.

    \subsection{Медный образец.}
    Будем измерять силу, действующую на образец в магнитном поле. Подвесим образец к весам,
    включим электромагнит и будем записывать показания.

\end{document}