\documentclass{article}
\usepackage[utf8]{inputenc}
\usepackage[russian]{babel}
\usepackage{graphicx}
\usepackage{wrapfig}
\usepackage{float}

\graphicspath{ { ./data/images/ } }
\author{Александр Романов Б01-107}
\date{}
\title{3.4.1 Диа- и парамагнетики}

\begin{document}
\maketitle

\section{Введение}

    \subsection{Цель работы}
    Измерение магнитной восприимчивости диа- и парамагнитного образцов.

    \subsection{В работе используется}
    Электромагнит, весы, милливеберметр, регулируемый источник постоянного тока, образцы.

\section{Работа}

    \begin{table}[H]
    \begin{tabular}{|c|c|}
        I, amp & Ф, mWb \\
        0,3  & 99     \\
        0,6  & 185,3  \\
        0,91 & 288,6  \\
        1,20 & 372,5  \\
        1,5  & 463    \\
        1,81 & 545    \\
        2,2  & 653,7  \\
        2,6  & 735,2  \\
        3,02 & 806,9 
    \end{tabular}
    \end{table}


\end{document}