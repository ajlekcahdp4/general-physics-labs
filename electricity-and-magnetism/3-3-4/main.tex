\documentclass{article}
\usepackage[utf8]{inputenc}
\usepackage[russian]{babel}
\usepackage{graphicx}
\usepackage{amsmath}
\usepackage{breqn}
\usepackage{wrapfig}
\usepackage{float}

\graphicspath{ {./data/images} }
\author{Александр Романов Б01-107}
\date{}
\title{3.3.4 Эффект Холла в проводниках}

\begin{document}
\maketitle
\section{Введение}
\subsection{Цель работы}
Измерение подвижности и конуентрации носителей заряда в проводниках.
\subsection{В работе используются} 
Электромагнит с регулируемым источником питания; вольтетр; амперметр; миллиамперметр;
милливебберметр; источник питания (1.5 V); Образец легированного германия.

\section {Работа}

\end{document}