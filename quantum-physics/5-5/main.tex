\documentclass{article}
\usepackage[utf8]{inputenc}
\usepackage[russian]{babel}
\usepackage{graphicx}
\usepackage{amsmath}
\usepackage{breqn}
\usepackage{wrapfig}
\usepackage{float}
\usepackage{multirow}
\usepackage{caption}
\usepackage{subcaption}

\graphicspath{ {./data/images} }
\author{Александр Романов Б01-110}
\date{}
\title{5.5.5 Компьютерная сцинтилляционная $\gamma$-спектроскопия}

\begin{document}
\maketitle
\section{Работа}
Включим компьютер и питание экспериментальной установки. Будем располагать измеряемые образцы в 
сцинтилляционном счётчике, запускать измерение и ждать пока на экране не появится чёткая картина пиков,
связанных с фотоэффектом, эффектом Комптона и т.д. Результаты сохраним.
\section{Обработка результатов}

\section{Выводы}
\end{document}
